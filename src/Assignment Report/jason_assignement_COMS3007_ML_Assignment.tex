\documentclass[11pt]{article}

\usepackage{fancyhdr}
 
\pagestyle{fancy}
\fancyhf{}
\lhead{University of the Witwatersrand}
\rfoot{School of Computer Science and Applied Mathematics}
\pagenumbering{roman}

\begin{document}

\begin{titlepage}

\newcommand{\HRule}{\rule{\linewidth}{0.3mm}} % Defines a new command for the horizontal lines, change thickness here
\renewcommand\section{\@startsection{section}{1}{\z@}%
                                  {-3.5ex \@plus -1ex \@minus -.2ex}%
                                  {2.3ex \@plus.2ex}%
                                  {\normalfont\large\bfseries}}

\center % Center everything on the page
 
%----------------------------------------------------------------------------------------
%	HEADING SECTIONS
%----------------------------------------------------------------------------------------

\textsc{\LARGE University of the Witwatersrand}\\[1.5cm] % Name of your university/college
\textsc{\Large School of Computer Science and Applied Mathematics}\\[0.5cm] % Major heading such as course name

%----------------------------------------------------------------------------------------
%	TITLE SECTION
%----------------------------------------------------------------------------------------

\HRule \\[0.4cm]
{ \huge \bfseries COMS3007: Machine Learning Assingment}\\[0.4cm] % Title of your document \\
  \large 13 May 2016
\HRule \\[1.5cm]
 
%----------------------------------------------------------------------------------------
%	AUTHOR SECTION
%----------------------------------------------------------------------------------------
\begin{minipage}{1\textwidth}
	\Large \emph By Chalom, J. (711985)\\
\end{minipage}


\vfill % Fill the rest of the page with whitespace

\end{titlepage}
\begin{page}
\section{Purpose}
Find a suitable dataset on which you can apply various supervised learning algorithms.
This can be either a classication problem or a regression problem.
Apply various supervised learning algorithms to your dataset.

\section{Dataset Used}
A description of your dataset: what are the attributes, what are the targets, how many
datapoints do you have, and some sample datapoints from the dataset. State what you are
trying to predict with the data.\\

\section{General Restrictions Applied}
A description of how you structured your inputs/targets and normalised the data, and the
split into training/validation/test data.\\

Square images\\
Real life / contains a certain amount of poission or normally distributed data (pixels)\\
All the same size\\

\section{Algorithms Used}
Neural Network With Perceptron Learning (Forward Propogation)\\
NN with Back Propogation\\
NN with RBF\\

\section{Methods Applied to The Problem}
A list of supervised learning algorithms you applied to the data, together with the details
of each implementation and the error on the test set.\\\\
For example:\\
Neural network with one hidden layer with 50 nodes,  = 0:3, error = ...\\
Neural network with two hidden layers with 20 nodes in rst layer and 15 in second,  = 0:2,
error = ...\\
RBF network with 25 RBF nodes and random centres,  = 0:25, error = ...\\
RBF network with 40 RBF nodes and centres obtained by k-means,  = 0:25, error = ...\\
etc.\\

\\


\section{Results}
A brief discussion of your results from the various algorithms. E.g., what worked best/worst
and why you think this is so.\\

Images of results before and after\\

\section{Conclusion}


\section{Discussion}

\end{page}
\end{document}
