\documentclass[11pt]{article}

\usepackage{fancyhdr}
 
\pagestyle{fancy}
\fancyhf{}
\lhead{University of the Witwatersrand}
\rfoot{School of Computer Science and Applied Mathematics}
\pagenumbering{roman}

\begin{document}

\begin{titlepage}

\newcommand{\HRule}{\rule{\linewidth}{0.3mm}} % Defines a new command for the horizontal lines, change thickness here
\renewcommand\section{\@startsection{section}{1}{\z@}%
                                  {-3.5ex \@plus -1ex \@minus -.2ex}%
                                  {2.3ex \@plus.2ex}%
                                  {\normalfont\large\bfseries}}

\center % Center everything on the page
 
%----------------------------------------------------------------------------------------
%	HEADING SECTIONS
%----------------------------------------------------------------------------------------

\textsc{\LARGE University of the Witwatersrand}\\[1.5cm] % Name of your university/college
\textsc{\Large School of Computer Science and Applied Mathematics}\\[0.5cm] % Major heading such as course name

%----------------------------------------------------------------------------------------
%	TITLE SECTION
%----------------------------------------------------------------------------------------

\HRule \\[0.4cm]
{ \huge \bfseries COMS3007: Machine Learning Assingment}\\[0.4cm] % Title of your document \\
  \large 13 May 2016
\HRule \\[1.5cm]
 
%----------------------------------------------------------------------------------------
%	AUTHOR SECTION
%----------------------------------------------------------------------------------------
\begin{minipage}{1\textwidth}
	\Large \emph By Chalom, J. (711985)\\
\end{minipage}


\vfill % Fill the rest of the page with whitespace

\end{titlepage}
\begin{page}
\section{Purpose}
\\Find a suitable dataset on which you can apply various supervised learning algorithms.
This can be either a classication problem or a regression problem.
Apply various supervised learning algorithms to your dataset.

My project involves image processing and its relation to machine learning. I went out to test different network methods, and find a network setup which works well with learning image filtering. The filter I chose to train these different methods with is specifically a filter, I created which reduces an image from RGB scale to a Black White scale where each pixel is reduced to a bw value based on its luminosity and its neighbours luminosity.

I created an image processor program which has the ability to convert images between formats such as jpeg, png and json. This program is also used for pre-processing the data such as normalising it and applying filters to be used as the target dataset. This program can also generate randomly generated 'noisey' images and scale or crop the images to the desired dimensions.

\section{Dataset Used}
\\A description of your dataset: what are the attributes, what are the targets, how many
datapoints do you have, and some sample datapoints from the dataset. State what you are
trying to predict with the data.\\

I have used data found from the Wikicommons public domain image repository. I have also created a set of random RGB images which are also going to be used to make sure that the networks do not learn regular features which may be present in the images chosen from Wikicommons. \\

The images chosen from online are all photographs which have either been scanned in or taken with a digital imaging device such as a CCDU or digital camera. This is to make sure that the data has even spread of pixels which do not have the same kind regularity as digital images may contain. This regularity could bias any trained network which will then provide results which are not desireable.

All the images are preprocessed to have the same square dimensions of 100 x 100 pixels, to get these dimensions the images are all cropped from the centre of the image. The target dataset is produced by applying the image filter to each input image.

The images are then converted to a filetype which can be used by the machine learning library. This is done by 'vectorising' the image or turning the two dimensional array of pixels into a one dimensional array of pixel values. TODO: add here

The aim of this project is to predict the change in pixels caused by the use of the bw filter and potentially improve on the filter. Optimise bw2 filter which is very slow?

\section{General Restrictions and Normalisation Applied}
\\A description of how you structured your inputs/targets and normalised the data, and the
split into training/validation/test data.\\

All the images are square and have the dimensions of 100 x 100 pixels. The images are either randomly generated or are photographs which have been digitally scanned from the real world. The images have been preprocessed by using the bw filter in order to become the target dataset.

Most of the training data will come from the images pull from the wikicommons repository. Validation and test data will have 10% real world images and the rest will come from randomly generated images. The rendomly generated images have better spread of pixel data and so are better for making sure bias in the network is kept to a minimum.

The output of the network was in the same form as the input into the network. This allows the output to be reconstructed into an image by the same image processing program described earlier.

Two strategies for data input was used. The first as described above was to input images into the network as seperate data values in the set. This used 100 x 100 images, for both the data inputs and targets. The inputs to the network each represented a pixel from an image. So for the data used 10000 input nodes and 10000 output nodes were used.

The other strategy was to have an image as the input dataset and rather than looking at the entire image as the input each pixel and its respective RGB values were the inputs to the network. This strategy had 3 input nodes and 3 output nodes with many more hidden layers and nodes than the first strategy.

Both strategies are compared below.

\section{Algorithms Used}
Neural Network With Perceptron Learning (Forward Propogation)\\
NN with Back Propogation\\
NN with RBF\\

Several methods were applied to this problem:
0. Neural network with back propogation
1. Multi-layer neural network
2. Multi-layer neural network with back propogation
3. RBF Neural Network

Within these networks several variations on parameters were used. (see the next section)

\section{Methods Applied to The Problem}
A list of supervised learning algorithms you applied to the data, together with the details
of each implementation and the error on the test set.\\\\
For example:\\
Neural network with one hidden layer with 50 nodes,  = 0:3, error = ...\\
Neural network with two hidden layers with 20 nodes in rst layer and 15 in second,  = 0:2,
error = ...\\
RBF network with 25 RBF nodes and random centres,  = 0:25, error = ...\\
RBF network with 40 RBF nodes and centres obtained by k-means,  = 0:25, error = ...\\
etc.\\

\\


\section{Results}
A brief discussion of your results from the various algorithms. E.g., what worked best/worst
and why you think this is so.\\

Images of results before and after\\

\section{Conclusion}


\section{Discussion}

\end{page}
\end{document}
